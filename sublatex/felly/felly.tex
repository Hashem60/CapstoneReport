\documentclass[../../main]{subfiles}
\begin{document}
\section{NAVIGATION}
    Autonomous navigation is a rapidly growing field and has become a
    pivotal area of research and application in robotics, with numerous
    applications in industries such as transportation, manufacturing, and
    warehousing. One of the fundamental tasks for autonomous robots is the
    ability to navigate their environment independently with minimal human
    intervention, that is a crucial factor or a key milestone for the
    development of intelligent systems capable of interacting with the real
    world. Among the various techniques used for autonomous navigation, line
    following remains one of the most essential and widely researched
    methods, especially for mobile robots, where the robot must track a
    specific line on the ground. While it may appear straightforward, line
    following involves a variety of challenges, including sensor
    calibration, precise control of the robot's motors, and handling of
    unpredictable environmental factors.For this project the primary
    objective was eventually designing and developing a simulated robot
    capable of accomplishing a specific task while following a line and
    avoiding obstacles and maintain accurate control over its movements
    within a simulated environment using the \textbf{Robot Operating System
    (ROS)}, which is widely used in both research and industry due to its
    flexibility, modularity, and wide range of libraries and tools. ROS
    provides a powerful framework for developing and simulating robotic
    applications.
    
    
    \section*{Methods used to design:}

    
    It is necessary outline the methodology used to design the robot, its
    navigation system, and the simulation environment. The method should
    cover both the hardware (even though it\textquotesingle s simulated) and
    software design, as well as the control algorithms used to ensure the
    robot follows the line autonomously.
    
    The design of the autonomous line-following robot involved a series of
    methodical steps to ensure the successful integration of hardware
    (simulated), software (ROS-based), and control algorithms to enable
    autonomous navigation. The following sections describe the approach
    taken to design and implement the robot and its navigation system in the
    simulation environment.
    
    \section{For Line Detection and Following:}
    
    \subsection{Camera-based Vision Method:}
    
    Using \textbf{cameras} for line-following is an advanced approach in
    robotics, where the robot uses visual information to detect and track a
    line or path. This technique, often referred to as \textbf{vision-based
    line following}, involves utilizing computer vision algorithms to
    process images captured by the robot\textquotesingle s camera and
    determine the robot\textquotesingle s position relative to the line. To
    use a camera for line-following, the camera is needed to be mounted on
    the robot, typically facing downward or slightly tilted to capture the
    surface on which the line is drawn which is the case with the robot used
    for this project.
    
    This process was accomplished through many steps via simulation such :
    

    \section*{Setting up the Simulation Environment}

    
    The simulation platform was set as with the following components:
    
    \begin{itemize}
    \item
      \textbf{Robot Model}: The robot in the simulation should have a camera
      mounted on it. The robot model could be a differential-drive robot, a
      mobile platform with wheels, or a more complex robot with additional
      sensors. \textbf{The robot model used is a differential-drive mobile
      robot.}
    \item
      \textbf{Camera Sensor}: Simulated cameras in the environment will
      capture images of the ground, where the line is located. The camera
      was set up to view downward, directly in front of the robot.
    \item
      \textbf{Line (Path)}: lines on the ground can be created in the
      simulation. The line can be straight, curved, or even follow a more
      complex path. For this project black lines were drawn within the world
      created in Gazebo.
    \item
      \textbf{Simulator}: The simulation platform used is:
    \end{itemize}
    
    \begin{quote}
    \textbf{Gazebo}: Commonly used with \textbf{ROS} (Robot Operating
    System) for robot control and visualization.
    \end{quote}
    
    
      \section*{Image Processing for Line Detection}
    
    
    \begin{quote}
    After setting up the simulated robot and camera, the robot will need to
    process the images captured by the camera to detect the line. The core
    part of this system involves \textbf{image processing} and
    \textbf{vision algorithms}.
    \end{quote}
    
    \begin{itemize}
    \item
      \textbf{Steps for Image Processing:}
    \end{itemize}
    
    \begin{itemize}
    \item
      \textbf{Capture Image from the Camera}:
    \end{itemize}
    
    \begin{quote}
    In the simulation, it is necessary to get access to the camera feed in
    real-time. With \textbf{ROS}, there can be subscriber to the
    camera\textquotesingle s image topic such
    \textbf{/camera/rgb/image\_raw}
    \end{quote}
    
    \begin{itemize}
    \item
      \textbf{Convert the Image to Grayscale}:
    \end{itemize}
    
    \begin{quote}
    \textbf{grayscale conversion} can be used to reduce the complexity of
    the image. Since the line will typically have a contrasting colour
    (black or white), working with a grayscale image makes it easier to
    detect the line.
    \end{quote}
    
    \begin{itemize}
    \item
      \textbf{Thresholding} will convert the grayscale image into a binary
      image where the line is white (or black) and the background is the
      opposite colour. You can use a simple threshold or adaptive
      thresholding for better results in varying lighting conditions.
    \end{itemize}
    
    \begin{itemize}
    \item
      \textbf{Detect the Line}:
    \end{itemize}
    
    \begin{quote}
    Once the image is binary, \textbf{edge detection}, \textbf{contour
    detection}, or \textbf{Hough Transform} can be used cto detect the line.
    For instance, you can use \textbf{Hough Line Transform} to detect
    straight lines. For the project they were combined and used together for
    line detection.
    \end{quote}
    
    \begin{itemize}
    \item
      \textbf{Determine the Robot's Position Relative to the Line}:
    \end{itemize}
    
    \begin{quote}
    \textbf{The centroid} of a detected line is calculated t hoo measure or
    estimate w far the robot is from the centre of the line in the camera
    frame.
    \end{quote}
    
    \subsection{1.2 Control Algorithm for Line Following}
    
    Once the line is detected, the robot needs to be controlled to follow
    it. The basic principle is to adjust the robot's steering based on its
    position relative to the line. PID Controller algorithm can be used to
    keep the robot following line in a steady way.
    
    The \textbf{PID (Proportional-Integral-Derivative)} controller is
    commonly used in line-following robots.
    
    \begin{itemize}
    \item
      \textbf{Proportional}: The steering correction is proportional to how
      far the robot is from the line's centre.
    \item
      \textbf{Integral}: This helps eliminate accumulated errors over time.
    \item
      \textbf{Derivative}: This predicts and reacts to the rate of change in
      the error (speed of deviation).
    \end{itemize}
    
    \subsection{1.3 Integrating the Line Following with the Simulator}
    
    Finally, the image processing and control algorithms needed to be
    integrated into the simulation environment. With ROS this integration is
    usually made through Gazebo. When \textbf{ROS} is used the robot cab ne
    set up with a simulated camera and subscribe to the camera feed. ROS
    provides libraries like CV BRIDGE to convert the ROS image messages into
    
    \section{For Building Map}
    
    \subsection{LIDAR}
    
    The map was built using SLAM (Simultaneous Localization and Mapping) in
    a simulation thanks to the lidar sensor. Using LIDAR (Light Detection
    and Ranging) with SLAM (Simultaneous Localization and Mapping) is a
    common and effective approach for building a map of an environment while
    simultaneously localizing the robot within that environment. In ROS,
    LIDAR is often used as the primary sensor for SLAM, especially for 2D
    SLAM algorithms like GMapping and Hector SLAM. The algorithm used for
    SLAM was GMapping in this project.The lidar can be used to avoid
    obstacles as well.
    
    \subsection{QR Code Scanning}
    
    QR Code Detection with OpenCV was used, the OpenCV's QRCodeDetector
    detects and decode QR codes in the robot\textquotesingle s camera feed.
    The QR code are used to help the robot to navigate autonomously, the QR
    code information to know its current position and calculate the closest
    path to the destination.
    
    After building the map , it will be saved for autonomous navigation
    purpose.
    
      \section{Realistic constraints}
    
    
    \section{computational-power-and-resources}    

    \subsection{cpu-and-gpu-limitations}

    \begin{itemize}
    \item
      \emph{CPU Limitations}: Simulations, especially those involving
      complex algorithms like SLAM or sensor fusion, can be CPU-intensive.
      The computational demand increases with the complexity of the
      robot\textquotesingle s tasks, such as real-time mapping,
      localization, or decision-making. A limited CPU performance could
      cause delays, lag, or even make real-time processing difficult.
    \item
      \emph{GPU Constraints}: If you\textquotesingle re using computer
      vision algorithms (e.g., QR code recognition, camera-based
      line-following), a GPU can greatly accelerate image processing.
      However, not all simulations leverage GPU power, and if the GPU is not
      sufficient or not utilized properly, the simulation could run slower
      or experience bottlenecks.
    \end{itemize}
    
    \subsection{memory-ram-constraints}   
    \begin{itemize}
    \item
      \emph{High Memory Usage}: Simulations that involve large
      environments, dense sensor data (such as LIDAR point clouds), or
      high-resolution images can require a significant amount of RAM. If the
      memory usage exceeds the available capacity, it can lead to slow
      performance, crashes, or even the inability to run the simulation at
      all.
    \item
      \emph{Data Storage for Sensor Data}: Storing large amounts of sensor
      data (e.g., from LIDAR, cameras, IMUs) generated during a simulation
      can strain the system's storage, especially if you need to log or save
      sensor outputs for later analysis. In a large-scale simulation, this
      could be a limiting factor.
    \end{itemize}
    
    \subsection{real-time-performance}
    
    \begin{itemize}
    \item
      \emph{Real-Time Simulation Constraints}: Achieving real-time
      performance in simulations can be difficult, especially when dealing
      with complex tasks such as real-time SLAM or path planning. The
      simulation might not run at the required frame rate (e.g., 30Hz or
      higher), leading to delays in robot control and decision-making.
    \item
      \emph{Simulation Time vs Real Time}: If the simulation is not
      running at real-time speed (1:1 with physical time), it can make it
      harder to validate time-sensitive behaviors. For example, a robot
      using SLAM may not update its map fast enough in a simulation that
      runs too slowly, affecting navigation and decision-making in
      real-world applications.
    \end{itemize}
    
    \section{complexity-of-the-simulated-environment}

    \subsection{size-and-detail-of-the-simulation}    
    \begin{itemize}
    \item
      \emph{Environmental Complexity}: Large and complex simulated
      environments with many dynamic obstacles, detailed textures, and
      various types of surfaces can be computationally expensive. More
      detailed environments require more processing power to render,
      simulate physics, and handle sensor data.
    \item
      \emph{Realistic Physics}: Physics engines in simulations (e.g.,
      Gazebo, V-REP) simulate the interactions between the robot and the
      environment, including forces like friction, gravity, and object
      collisions. While necessary for realistic testing, these physics
      simulations can be computationally demanding, especially in dynamic
      environments with many objects.
    \end{itemize}
    
    \subsection{dynamic-objects-and-movements}    
    \begin{itemize}
    \item
      \emph{Real-Time Object Simulation}: When simulating environments
      with moving objects (e.g., pedestrians or vehicles), the computation
      required to simulate their motion and interactions with the robot can
      add significant overhead. This becomes particularly challenging if
      multiple objects are moving in complex patterns at the same time.
    \end{itemize}
    
    \section{simulating-sensor-data}

    \subsection{a.-sensor-data-processing-load}
    \begin{itemize}
    \item
      \emph{LIDAR and Point Cloud Data}: LIDAR sensors generate large
      amounts of data, especially in 3D environments. Processing the point
      cloud data in real time requires significant computing resources,
      particularly when applying algorithms like SLAM to build and update
      maps. The more data points and the larger the map, the more processing
      power is required.
    \item
      \emph{Camera Data for Visual Processing}: Cameras generate
      high-resolution images that need to be processed for tasks like QR
      code detection or line-following. Processing these images (especially
      with advanced computer vision techniques such as feature detection or
      deep learning) can be computationally expensive. Depending on the
      image resolution and the complexity of the detection algorithms, this
      could put a strain on the available computing resources.
    \end{itemize}
    
    \subsection{real-time-sensor-fusion}
    
    \begin{itemize}
    \item
      \emph{Combining Data from Multiple Sensors}: If you are fusing data
      from multiple sensors (e.g., combining LIDAR, IMU, camera data for
      SLAM), the computational load increases significantly. Sensor fusion
      algorithms, such as Kalman Filters or particle filters, need to
      process data from all sensors in real time, which may not be feasible
      with limited computational resources.
    \end{itemize}
    
    \section{algorithmic-constraints}
    
    \subsection{a.-computational-cost-of-slam}

    \begin{itemize}
    \item
      \emph{SLAM Algorithm Complexity}: Algorithms used for SLAM (like
      Extended Kalman Filters, GraphSLAM, or particle filters) are
      computationally expensive, especially when the robot has to map a
      large area in real-time. As the environment grows, the number of
      calculations needed to maintain and update the map increases,
      potentially leading to slower performance or the need for more
      powerful hardware.
    \item
      \emph{Map Size and Resolution}: The higher the resolution and
      accuracy of the map, the more computational resources are required to
      store and update it. Large maps with high-detail features demand
      significant memory and processing power to handle updates, store the
      data, and process localization.
    \end{itemize}
    
    \subsection{b.-path-planning-and-decision-making}
    \begin{itemize}
    \item
      \emph{Complex Path Planning Algorithms}: Path planning algorithms,
      such as A*, RRT, or D* algorithms, may require substantial
      computational resources depending on the complexity of the
      environment. If your robot is navigating a large or cluttered
      environment, calculating collision-free paths in real-time becomes a
      challenge, especially if there are many dynamic obstacles to avoid.
    \item
      \emph{Decision-Making Algorithms}: When the robot makes decisions
      based on sensor input, running machine learning algorithms or decision
      trees to determine the next action (e.g., go towards a QR code or
      follow a line) can also be computationally expensive. Depending on the
      complexity of the logic, this could limit the speed and responsiveness
      of the simulation.
    \end{itemize}
    
    \section{simulation-software-limitations}
    
    \subsection{simulation-engine-efficiency}
    \begin{itemize}
    \item
      \emph{Performance of the Simulation Engine}: Different simulation
      platforms (Gazebo, V-REP, Webots, etc.) have varying levels of
      efficiency. Some simulators might be highly optimized for certain
      types of robots or algorithms, while others may be slower or require
      more resources to simulate complex systems. Choosing the right
      simulation platform for your application is critical for balancing
      performance and accuracy.
    \item
      \emph{Plugin Overhead}: Many simulators rely on plugins for
      additional functionality (e.g., camera sensors, LIDAR). The more
      plugins you add or the more complex the plugin behavior, the more
      computational overhead is introduced. This can affect simulation
      performance, especially in large-scale environments or real-time
      applications.
    \end{itemize}
    
    \subsection{parallelization-and-multithreading}
    \begin{itemize}
    \item
      \emph{Limited Parallel Processinh}: Not all simulation environments
      or algorithms are well-optimized for parallel processing, which means
      that tasks may not fully utilize multi-core CPUs or GPUs. For example,
      in simulations involving real-time SLAM, the ability to parallelize
      sensor data processing or SLAM computation can significantly reduce
      the computational load, but not all algorithms or environments support
      this effectively.
    \item
      \emph{Threading Issues}: Running multiple processes (e.g., sensor
      readings, actuator control, SLAM) in parallel in a simulation may
      introduce issues such as deadlock or race conditions if not properly
      managed. This can lead to slower simulation times or unresponsive
      behavior in your robot model.
    \end{itemize}
    
    \section{simulating-real-time-constraints}
    
    \subsection{real-time-control-vs.-simulation-speed}
    \begin{itemize}
    \item
      \emph{Synchronization Issues}: Achieving true real-time performance
      is difficult in a simulation. Many simulations allow you to adjust the
      time scale (e.g., running the simulation faster than real time for
      testing), but this introduces the issue that the control algorithms
      might not have the same timing constraints they would in a real robot.
      Ensuring that the simulation runs with a fixed time step, synchronized
      with the robot's control loops, is essential but can be
      computationally demanding.
    \item
      \emph{Hardware-in-the-loop (HIL) Simulation}: When integrating real
      hardware with a simulator (e.g., for testing SLAM on real sensors),
      the latency and real-time computation required can cause performance
      bottlenecks. This is especially true if communication between the
      simulation and physical robot hardware is slow or not synchronized
      properly.
    \end{itemize}

\end{document}