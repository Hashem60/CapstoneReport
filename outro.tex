
\documentclass[main]{subfiles}
\begin{document}

% \newpage
% \section{Conclusion}

% It should not only be functional, but also practical, manufacturable,
% and economic, taking into consideration all the constraints. The
% literature survey focuses on the fact that the gearbox-equipped drive
% wheel system is a main concern in the design of an AGV. Material
% selection, load-carrying capacity, torque optimization, and
% manufacturability are the key factors to be considered. In this paper,
% design and simulation for performance improvement of the AGV using
% SolidWorks are proposed to overcome these bottlenecks and help in the
% development of reliable high-performance drive systems for AGVs.

% The middle drive wheels with an integrated gearbox were designed and
% further validated for use in an AGV. The system was reliable, efficient,
% and easy to manufacture. Future work can be the physical development of
% the prototype and performance testing at site applications.

This project has successfully addressed the design and optimization of critical components for automated systems, focusing on bearing wheels, caster wheels, and their integration into advanced material handling solutions such as AGVs (Automated Guided Vehicles). The study employed a systematic approach to ensure structural integrity, operational efficiency, and compliance with industry standards. Key achievements include the validation of the single-level scissor lift mechanism for a load capacity of 200 kg, the development of a robust chassis structure capable of supporting up to 300 kg (2943N), and the selection of the NSK 6201 bearing for its high performance and reliability in handling both radial and axial loads. SolidWorks was utilized for 3D modeling and finite element analysis (FEA), enabling accurate identification of potential failure points and enhancing structural integrity while minimizing prototyping costs. Through careful consideration of material properties, mechanical constraints, and environmental factors, the final designs demonstrate excellent durability, efficiency, and compliance with ISO 3691-4:2020 guidelines.

\section{Methodology and Design Approach}
The design process adopted a comprehensive methodology that combined theoretical calculations, empirical analysis, and digital tools. Extensive structural analysis and stress testing were conducted to validate the performance of the scissor lift mechanism and chassis, ensuring that the designs met specified operational requirements while maintaining necessary safety margins. SolidWorks simulations played a pivotal role in refining the design. FEA helped identify weak points in the assemblies, allowing for iterative improvements before physical prototyping. This approach not only enhanced the efficiency of the design process but also reduced costs significantly. Adherence to ISO 3691-4:2020 guidelines ensured that the bearing and caster wheels met stringent safety requirements, including braking systems, speed control, load handling, and stability. Compliance with these standards guarantees safe and efficient operation in industrial environments.

\section{Contributions to Industrial Applications}
The successful development of these components represents a significant advancement in automated material handling solutions. By integrating advanced safety features, control systems, and predictive maintenance capabilities, the system demonstrates improved reliability and functionality. Enhanced operational stability and structural integrity were achieved through meticulous load distribution and stress analysis. Improved load-handling capabilities for industrial trucks were realized, aligning with modern trends toward automation and sustainability. These contributions position the system as a reliable solution for industrial applications, addressing the growing demand for efficient and durable material handling equipment.

\section{Future Directions}
While the current project lays a strong foundation, several areas have been identified for future development to further enhance performance, safety, and sustainability. Future work will focus on optimizing the weight and strength of components by exploring lightweight materials. Modular attachment systems and enhanced payload configurations will also be investigated to improve versatility. The integration of smart sensors for real-time load monitoring and predictive maintenance is a priority, enabling proactive maintenance, reducing downtime, and extending the service life of components. Advanced emergency stop systems and collision avoidance capabilities will be developed to ensure the AGV system adapts to complex industrial environments while maintaining high safety standards. Exploring eco-friendly material alternatives aligns with global environmental regulations and industry trends, reducing the carbon footprint while enhancing durability and recyclability. Experimental testing and field applications are essential to validate the proposed designs in real-world scenarios, providing valuable insights into performance under varying conditions and helping refine the system further.

\section{Conclusion}
In conclusion, this project has demonstrated a structured and comprehensive approach to the design and analysis of bearing wheels, caster wheels, and AGV systems. By leveraging advanced tools like SolidWorks for simulation and adhering to industry standards such as ISO 3691-4:2020, the study provides a robust framework for developing high-performance, durable components. The successful implementation of the proposed designs marks a significant step forward in automated material handling solutions. With continued advancements in materials, smart technologies, and sustainable practices, these systems will evolve to meet emerging industrial demands, offering greater efficiency, versatility, and safety. Ultimately, this project paves the way for more reliable and innovative solutions in industrial automation and material transport.

\end{document}